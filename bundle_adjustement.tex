\documentclass{article}

\usepackage{amsfonts}
\usepackage{amsmath}
%\usepackage{hyperref}
%\usepackage{url}

\oddsidemargin = 18pt
\topmargin     = 0pt
\textwidth     = 450pt
\textheight    = 650pt
\linespread{1.25}


\begin{document}

\title{Bundle Adjustment Based Multi-Camera Rig Calibration}

\author{Sileye Ba}

\date{June 29th 2016}

\maketitle

\section{Introduction}

This document presents the bundle adjustment solution to the calibration of a synchronized multi-camera rig.

We consider a multi-camera rig composed of $K$ cameras. Every camera $k$ has intrinsic parameters $\theta_k = (f_k,c_k,d_k)$ (focal, optical center, distortion). 

We denote the rigid motion to map a point in the world coordinate system to the $k$'th camera coordinate system as ${}^k\mathcal{T}_w =[{}^kR_w\;{}^kT_w]$, where ${}^kR_w$ is a rotation matrix, and ${}^kT_w$ a translation vector.

\section{The Bundle Adjustment Problem}
The camera rig has acquired t=1,...,T views. At each time t, camera k, acquires a view $I_{kt}$. 
%
At each time $t$, a number of $N_{kt}$ keypoint, $x_{nkt},\; n=1,..., N_{kt}$ are detected on view $I_{kt}$.

Using keypoints descriptor\footnote{In our case we use AKAZE descriptors.}, keypoints $x_{nkt}$ on camera $k$, can be matched to keypoints $x_{mlt}$ on camera $l$, with a  matching weights $w_{nmklt}$\footnote{In a hard assignment matching situation, which is our case,  $w_{nmklt}$ is either 0 or 1.}. This means that, with confidence $w_{nmklt}$,  $x_{nkt}$ and $x_{mlt}$ represent the projection of the same 3D world's point.

We denote ${}^w X_{nmt}$, the 3D point in the world coordinate system that projects as  $x_{nkt}$ onto camera $k$'s and as  $x_{mlt}$ onto camera $l$'s view. This is formally written as  
\begin{align}\label{eq:direct-projection}
x_{nkt} = \Pi(\theta_k, {}^k\mathcal{T}_w {}^wX_{nmt}) \\ \nonumber
x_{mlt} = \Pi(\theta_l, {}^l\mathcal{T}_w {}^wX_{nmt})
\end{align}
where $\Pi(\theta_k,{}^k\mathcal{T}_w .)$ is the projective transformation mapping 3D points represented into the world referential, into 2D points on the $k$'th camera image plane.

Thus, using relations in equation \eqref{eq:direct-projection}, to find the camera intrinsic parameters $\theta$, the 3D world to cameras transforms $\mathcal{T}_w$, and the 3D world's points ${}^wX$, the bundle adjustment problem can be stated as finding the arg-max of the cost function:
\begin{equation}\label{eq:static-BA}
J(\theta,\mathcal{T}_w,{}^wX) = \sum_{t} \sum_{k \neq l} \sum_{n,m}  w_{nmklt}\left(||x_{nkt} - \Pi(\theta_k, {}^k\mathcal{T}_w {}^wX_{nmt})||_2^2 +||x_{mlt} - \Pi(\theta_l, {}^l\mathcal{T}_w {}^wX_{nmt})||_2^2  \right)
\end{equation}
where $||.||_2$ is the $\mathbb{R}^2$ Euclidean distance.

Let us denote by $\hat{x}_{nkt} = \Pi(\theta_k, {}^k\mathcal{T}_w {}^wX_{nmt})$, the image point obtained by projecting the 3D point $X_{nmt}$. %
%This image point depends on the intrinsic parameters $\theta_k$, the rigid motion (change basis) parameter ${}^k\mathcal{T}_w$, and the 3D point $X_{nmt}$. 
%
Using this notation, the bundle adjustment problem can be written as

\begin{equation}\label{eq:static-BA-2}
J(\theta,\mathcal{T}_w,{}^wX) = \sum_{t} \sum_{k \neq l} \sum_{n,m}  w_{nmklt}\left(||x_{nkt} - \hat{x}_{nkt}||_2^2 +||x_{mlt} - \hat{x}_{mlt}||_2^2  \right)
\end{equation}


\section{Solving The Bundle Adjustment Problem}

The bundle adjustment,the BA problem requires finding the parameters values minimizing problem \eqref{eq:static-BA-2}. This can be done by finding parameters vanishing the gradient of cost function $J(\theta,\mathcal{T}_w,{}^wX)$ which involves three main terms: $\nabla_{\theta_k}J$, $\nabla_{\omega_k}J$ where $\omega_k$ is the special Euclidean  rigid motion parametrization ${}^k\mathcal{T}_w$ , and $\nabla_{X_{nmt}}J$.

If we denote by 
$
J_{nm}(\theta_k,\omega_k,X_{nmt}) = ||x_{nkt} - \hat{x}(\theta_k,\omega_k,X_{nmt})||_2^2
$
we need to compute differential of $J_{nm}$ suffices to obtain the differential of $J$.

\section{Differentials of $J_{nm}$}

Differential of $J_{nm}$ can is defined as:
\begin{equation}\label{eq:differential-intrinsics-formula}
\lim_{\beta \rightarrow 0} \frac{J_{nm}(\theta_k,\omega_k,X_{nmt}) + J_{nm}(\theta_k+\beta d\theta_k,\omega_k ,X_{nmt})}{\beta} = <\nabla_{\theta_k}J_{nm}, d\theta_k>
\end{equation}
where $<,>$ is a scalar product. 
%
Assuming available a first order approximation of $\hat{x}(\theta_k +\beta d\theta_k,\omega_k,X_{nmt})$ around $\hat{x}(\theta_k,\omega_k,X_{nmt})$ as 
\begin{equation}
\hat{x}(\theta_k +\beta d\theta_k,\omega_k,X_{nmt}) \approx \hat{x}(\theta_k,\omega_k,X_{nmt}) + \beta \frac{\partial \hat{x}(\theta_k,\omega_k,X_{nmt})}{\partial \theta_k}d\theta_k +o(\beta^2)
\end{equation}\label{eq:projection-intrinsics-first-order}
where $\frac{\partial \hat{x}(\theta_k,\omega_k,X_{nmt})}{\partial \theta_k}$ is the Jacobian of $\hat{x}(\theta_k,\omega_k,X_{nmt})$ w.r.t. $\theta_k$.
Using differential formula in \eqref{eq:differential-intrinsics-formula} and the first order approximation in \eqref{eq:projection-intrinsics-first-order}, the differential $\nabla_{\theta_k}J_{nm}$ is
\begin{equation}
\nabla_{\theta_k}J_{nm} = \frac{\partial \hat{x}(\theta_k,\omega_k,X_{nmt})}{\partial \theta_k}^{\top}(x_{nkt} - \hat{x}(\theta_k,\omega_k,X_{nmt}))
\end{equation}

Following similar reasoning as previously, the other differential can be obtained as:
\begin{align*}
\nabla_{\omega_k}J_{nm} &= \frac{\partial \hat{x}(\theta_k,\omega_k,X_{nmt})}{\partial \omega_k}^{\top}(x_{nkt} - \hat{x}(\theta_k,\omega_k,X_{nmt})) \\
\nabla_{X_{nmt}}J_{nm} &= \frac{\partial \hat{x}(\theta_k,\omega_k,X_{nmt})}{\partial X_{nmt}}^{\top}(x_{nkt} - \hat{x}(\theta_k,\omega_k,X_{nmt}))
\end{align*}
Computing differential requires the derivation of Jacobians $ \frac{\partial \hat{x}(\theta_k,\omega_k,X_{nmt})}{\partial \theta_k}$, $\frac{\partial \hat{x}(\theta_k,\omega_k,X_{nmt})}{\partial \omega_k}$, $\frac{\partial \hat{x}(\theta_k,\omega_k,X_{nmt})}{\partial X_{nmt}}$.


%%%%%%%%%%%%%%%% APPENDICES

\appendix

\section{Projection of 3D world points onto the $k$'th camera image plane $\hat{x}_{k} = \Pi(\theta_k, {}^k\mathcal{T}_w {}^wX)$}
The principle of the projection model is the following. 

First the 3D world, using the world to camera mapping, point ${}^wX$ is transformed into the camera reference as 
$${}^kX={}^k\mathcal{T}_w {}^wX =\mathcal{T}(\omega_k) {}^wX.$$

Then, denoting the Euclidean distance in the suitable space  as $||.||$, the camera point is mapped to the camera unit sphere as 
$${}^sX = \frac{{}^kX}{||{}^kX||}.$$

Then the point of the camera unit sphere translated by unity in the z axis direction, and projected onto the camera image plane as 
$$
{}^k\tilde{\mathbf{x}} = 
\left(
\begin{array}{c}
\frac{{}^sX^1}{{}^sX^3+1}\\▼
\frac{{}^sX^2}{{}^sX^3+1}\\
1
\end{array}
\right)
$$

If we define $r = ||{}^k \mathbf{x}||$,  the radially distorted projection image point is given by
$$ {}^d \mathbf{x} = g({}^k \mathbf{x}, \mathbf{k}) = \left(\sum_{n=1}^3 k_n r^n + 1 - \sum_{n=1}^3 k_n \right) {}^k \mathbf{x}.$$

If $\mathbf{f}=(f_1,f_2)$ denotes the camera focal, and $\mathbf{c}=(c_1,c_2)$ its principal point, and the intrinsic camera matrix defined as 
$$
\mathbf{K} =
\left(
\begin{array}{ccc}
f_1 & 0   & c_1\\
0   & f_2 & c_2\\
0   & 0   & 1
\end{array}
\right), 
$$
the image point in pixel is given as
$$
\tilde{y}_k = \mathbf{K} {}^d \tilde{\mathbf{x}}.
$$
%
The full projection model can be written as
$$
\tilde{y}_k(\mathbf{f},\mathbf{c},\mathbf{k},\omega_k,{}^wX) = \mathbf{K}(\mathbf{f},\mathbf{c}) \left(\sum_{n=1}^3 k_n r({}^k \mathbf{x}(\omega_k,{}^wX))^n + 1 - \sum_{n=1}^3 k_n \right) {}^k \mathbf{x}(\omega_k,{}^wX)
$$

\bibliographystyle{plain}
\bibliography{../biblio/biblio}
\end{document}